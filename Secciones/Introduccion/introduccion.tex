
\subsection{Contexto}
En la era digital, la generación de Modelos de Terreno Digital (MTD) es fundamental para diversos campos, incluyendo proyectos catastrales, simulaciones fluviales y gemelos digitales. La precisión de estos modelos depende en gran medida de la calidad de los datos de entrada, que suelen provenir de nubes de puntos obtenidas mediante tecnologías como LIDAR. Sin embargo, cuando la densidad de estos puntos es muy alta, los algoritmos tradicionales pueden perder fiabilidad, afectando la precisión del MTD resultante.


\subsection{Antecedentes}
Diversos algoritmos han sido desarrollados para la generación de MTDs, incluyendo métodos de interpolación, triangulación y segmentación. No obstante, estos métodos enfrentan desafíos significativos cuando se aplican a nubes de puntos de alta densidad. La capacidad de clasificar correctamente estos puntos es crucial para mantener la precisión del modelo final. Las redes neuronales han mostrado un gran potencial en tareas de clasificación y pueden ofrecer una solución prometedora para este problema.

\subsection{Problematica}
El desafío principal es desarrollar un algoritmo que pueda clasificar eficientemente una nube de puntos con alta densidad sin sacrificar la precisión. La alta densidad de puntos presenta dificultades tanto en términos de procesamiento como de exactitud, ya que los algoritmos existentes a menudo no pueden manejar el volumen de datos sin comprometer su desempeño.

