\section{Marco Teórico}

\subsection{Antecedentes}
La tecnología LiDAR ha evolucionado significativamente desde sus inicios, revolucionando la forma en que se capturan y analizan los datos geoespaciales. Inicialmente utilizada en aplicaciones militares, su uso se ha extendido a diversos campos, como la topografía, la arqueología y la gestión de recursos naturales. Los modelos digitales de terreno, por su parte, han pasado de ser representaciones básicas del terreno a modelos altamente precisos gracias a los avances en la resolución y en las técnicas de procesamiento de datos.
\espacio

\subsection{Conceptos Clave}
LiDAR, o Light Detection and Ranging, es una tecnología que utiliza pulsos de luz láser para medir distancias y crear mapas tridimensionales de la superficie terrestre. Las nubes de puntos son conjuntos de datos tridimensionales generados por escaneos LiDAR, que contienen información sobre la posición y altura de cada punto. Los modelos digitales de terreno (MTD) son representaciones tridimensionales del terreno que excluyen estructuras sobre la superficie, mientras que las redes neuronales son sistemas de aprendizaje automático que imitan el funcionamiento del cerebro humano para procesar datos complejos y encontrar patrones.
\espacio

\subsection{Teorías Relevantes}
La teoría del procesamiento de datos geoespaciales se centra en cómo se pueden recolectar, procesar y analizar datos espaciales para obtener información útil. En el contexto de la inteligencia artificial, las redes neuronales y el aprendizaje profundo son teorías que han revolucionado el análisis de grandes volúmenes de datos. Las redes neuronales, inspiradas en la estructura del cerebro humano, son particularmente efectivas en la clasificación y el reconocimiento de patrones, lo que las hace ideales para mejorar la precisión de los modelos digitales de terreno.
\espacio

\subsection{Estudios Previos}
Numerosos estudios han investigado el uso de LiDAR para la creación de modelos digitales de terreno. Por ejemplo, Smith et al. (2018) demostraron que el uso de LiDAR aéreo puede proporcionar modelos de alta resolución que son esenciales para la planificación urbana. Por otro lado, investigaciones recientes han explorado el uso de redes neuronales para mejorar la clasificación de nubes de puntos LiDAR, como el estudio de Zhang y Wang (2020), que mostró una mejora significativa en la precisión de los modelos al integrar técnicas de aprendizaje profundo.
\espacio

\subsection{Resumen}
En resumen, el marco teórico proporciona una base sólida para entender la importancia de la tecnología LiDAR, los modelos digitales de terreno y las redes neuronales en el contexto de esta investigación. La revisión de la literatura ha demostrado la necesidad de mejorar la precisión de los MTD y ha identificado las redes neuronales como una solución prometedora. Este marco teórico guía las metodologías adoptadas en este estudio, asegurando que las técnicas y enfoques utilizados estén bien fundamentados en investigaciones previas y teorías establecidas.